\chapter*{Введение}
\addcontentsline{toc}{chapter}{Введение}

В мире происходит научно-техническая революция, обусловленная гигантским скачком в достижениях науки и техники, в жизни всего общества. Развитие ракетно-космической техники, космические исследования и освоение космического пространства являются одним из характерных проявлений современной научно-технической революции. Космонавтика --- синтез того, что достигнуто сейчас мировой наукой и техникой \cite{actuality}.

Космические исследования, такие как разработка и создание ракетно-космических систем, работающих в космосе, искусственных спутников Земли, пилотируемых космических кораблей, и межпланетных автоматических станций, ускорили развитие некоторых научно-технических областей, которые до этого не были связаны непосредственно с космосом.

Основной и наиболее важной областью исследований сегодня является околоземное космическое пространство. Вслед за первыми искусственными спутниками были созданы и выведены на орбиты вокруг Земли тысячи других, имеющих разнообразное назначение и применение.

Цель летней практики --- разработка программного комплекса для сбора информации об искусственных спутниках Земли.

Для достижения указанной выше цели были выделены следующие ключевые задачи:
\begin{itemize}
	\item на основе данных о видимых искусственных спутниках Земли необходимо решить задачу планирования порядка получения информации о заданном объекте на протяжении суток в каждую секунду времени;
	\item при множественном выборе в планировании выделить систему приоритетов на основе весовых коэффициентов и оценке физического положения прибора наблюдения в конкретный момент времени;
	\item после сбора данных необходимо составить таймлайн объекта;
	\item следует выполнить обработку изображений на предмет шумов и осуществить повышение разрешения с целью улучшения визуальных характеристик объекта. При реализации этого функционала использовать математические модели, а также нейросети;
	\item также предложено реализовать собственную нейронную сеть, решающую задачу классификацию спутников по их снимкам, полученным благодаря специальному оборудованию мониторинга космического пространства.
\end{itemize}

Индивидуальным заданием является проектирование и создание программного обеспечения, позволяющего улучшить качество изображений спутников посредством удаления шумов и повышения разрешения, в рамках которого следует решить нижеперечисленные задачи:

\begin{itemize}
	\item изучить общие принципы повышения качества изображения;
	\item выполнить классификацию существующих методов;
	\item рассмотреть математические методы;
	\item изучить актуальные решения, использующие нейронные сети;
	\item реализовать дискретный модуль разрабатываемого программного комплекса;
	\item протестировать разработанное ПО.
\end{itemize}




