
%%%%%%%%%%%%%%%%%%%%%%%%%%%%%%%%%%%%%%%%%%%%%%
\chapter{Общие сведения}

\section{Характеристика организации}

Межгосударственная акционерная корпорация <<Вымпел>> --- ведущее предприятие российской оборонной промышленности в области РКО. Входит в состав АО <<Концерн <<ВКО <<Алмаз-Антей>> \cite{macvympel}.

Корпорация <<Вымпел>> отвечает за широкий комплекс наукоемких работ, связанных с проектированием, созданием, испытаниями и развитием систем, решающих задачи предупреждения о ракетном нападении (СПРН), противоракетной обороны (ПРО) и контроля космического пространства (СККП), создает и совершенствует программно-алгоритмическое обеспечение для одновременной обработки гиперобъемной информации и визуализации ее результатов на командных пунктах этих систем. 

Все системы и средства РКО работают в полностью автоматическом режиме, в реальном масштабе времени, с возможностью одновременного управления с командных пунктов.

\section{Техническое задание}

Индивидуальным заданием является проектирование и создание ПО, позволяющего улучшить качество изображений спутников посредством удаления шумов и повышения разрешения. 

Все необходимые данные были предоставлены организацией.


%%%%%%%%%%%%%%%%%%%%%%%%%%%%%%%%%%%%%%%%%%%%%%
\chapter{Классификация существующих методов повышения качества изображения}

\section{Методы фильтрации шумов}

Фильтрация --- подавление шума целевого изображения при максимальном сохранении деталей. Это незаменимая операция при предварительной обработке. Качество данного этапа напрямую влияет на эффективность анализа последующей обработки и анализа изображения. Обычно энергия сигнала сосредоточена в низком и среднем диапазоне амплитудного спектра, тогда как в более высоких частотных диапазонах интересующая информация часто перегружена шумом. Следовательно, фильтр, который может уменьшить амплитуду высокочастотных компонентов, может уменьшить влияние шума \cite{genfiltering}.

Выделяют два основных требования для обработки фильтрации:
\begin{itemize}
	\item не повреждать контур и края объектов;
	\item сделать изображение четким и качественным.
\end{itemize}

Далее представлены основные методы фильтрации шумов и описаны их принципы работы.
 


\subsection{Метод среднего арифметического}

Метод среднего арифметического состоит в том, что в заданной окрестности среднее значение всех пикселей является окончательным результатом вычисления, а вес каждого пикселя одинаков, что является обратной величиной общего количества пикселей. Среднее сглаживание является линейным. После задания параметров шаблон определяется и не будет изменяться из-за различных положений и распределения пикселей. Основной операцией линейного шаблона является свертка \cite{meanfilter}.


\subsection{Метод медианной фильтрации}

Медианный фильтр $M$ из входящего сигнала $C$ создает медианный образ сигнала $\widetilde{C}$.
Входящий сигнал $C$ подается на медианный фильтр $M:C \rightarrow \widetilde{C}$ \cite{medianfilter}.

В медианном фильтре сначала производится выбор значений, попавших в окно фильтра при нахождении окна в точке $x$, $\hat{O}(x):C \rightarrow O$.

Далее производится сортировка значений окна $O$ функцией сравнения значений $\Phi$ и строится упорядоченное множество $\Phi(O) \rightarrow \widetilde{O}$, а после выбирается медианное значение\footnote{Медиана набора чисел --- число, которое находится в середине этого набора, если его упорядочить по возрастанию, то есть такое число, что половина из элементов набора не меньше него, а другая половина не больше}: $m(\widetilde{O}) \rightarrow o_{m}$ и записывается в $\widetilde{C}(x)= o_{m}$.

Таким образом, медианный фильтр $M:C \rightarrow \widetilde{C}$ является последовательностью трех действий:
\begin{enumerate}
	\item Выбор значений, попавших в окно фильтра $\hat{O}(x):C \rightarrow O$.
	\item Сортировка значений окна $\Phi(O) \rightarrow \widetilde{O}$.
	\item Выбор из $\widetilde{O}$ медианного значения $m(\widetilde{O}) \rightarrow o_{m}$ и запись его в медианный образ сигнала $\widetilde{C}$ в точку с координатой $x$, $\widetilde{C}(x) = o_{m} $.
\end{enumerate}

Следует отметить, что фильтрационное окно может быть произвольной геометрической формы.


\subsection{Метод билатеральной фильтрации}

Билатеральный фильтр определяется следующим образом \cite{bilateralfilter}:

\begin{equation}
	I^\text{filtered}(x) = \frac{1}{W_p} \sum_{x_i \in \Omega} I(x_i)f_r(\|I(x_i) - I(x)\|)g_s(\|x_i - x\|),
\end{equation}

\begin{equation}
	W_p = \sum_{x_i \in \Omega}{f_r(\|I(x_i) - I(x)\|)g_s(\|x_i - x\|)},
\end{equation}
где
\begin{itemize}
	\item $I^\text{filtered}$ ---  изображение, после применения фильтрации;
	\item $I$---  исходное изображение;
	\item $x$ ---  координаты текущего пикселя для фильтрации;
	\item $\Omega$ ---  окно с центром в $x$, тогда $x_i \in \Omega$ соседний пиксель;
	\item $f_r$ ---  интенсивности пикселей;
	\item $g_s$ --- функция Гаусса.
\end{itemize}

Пиксель просто заменяется взвешенным средним его соседей.

\begin{equation}
	w(i, j, k, l) = \exp\left(-\frac{(i - k)^2 + (j - l)^2}{2 \sigma_d^2} - \frac{\|I(i, j) - I(k, l)\|^2}{2 \sigma_r^2}\right),
\end{equation}
где $\sigma_d$ и $\sigma_r$ --- сглаживающие параметры, и $I(i, j)$ и $I(k, l)$ --- интенсивности пикселей $(i, j)$ и $ (k, l)$ соответственно.

После вычисления весов, необходимо нормализовать их:
\begin{equation}
	I_D(i, j) = \frac{\sum_{k, l} I(k, l) w(i, j, k, l)}{\sum_{k, l} w(i, j, k, l)},
\end{equation}
где $I_D$ интенсивность пикселя $(i, j)$ без шума.

\section{Методы обратной свертки}

Обратная свертка, деконволюция, развертка --- операция, обратная свертке сигналов. Обратная свертка широко используется в обработке сигналов и изображений, а также для других инженерных и научных приложений \cite{deconvolution}.

В общем случае целью деконволюции является поиск решения уравнения свертки, заданного в виде:

\begin{equation}
	f*g=h
\end{equation}


Обычно $h$ --- записанный сигнал, а $f$ --- сигнал, который требуется восстановить, причем известно, что сигнал $h$ получен путем свертки сигнала $f$ с некоторым известным сигналом $g$. Если сигнал $g$ неизвестен заранее, его требуется оценить. Обычно это делается с помощью методов статистического оценивания.


\subsection{Обратная свертка Винера}

Ошибка системы равна разности между оценкой $d(t)$ и реальным значением $y(t)$ полезного сигнала $e(t)=d(t)-y(t)$ \cite{wd}. Минимальная среднеквадратическая ошибка по определению равна:

\begin{equation*}
	\eta=\overline{e^{2}}=\overline{d^{2}}-2\,\overline{dy}+\overline{y^{2}} =
\end{equation*}

\begin{equation*}
\overline{d^{2}}-2\int_{-\infty}^{+\infty} w(\tau)\overline{f(t-\tau)d(t)} \, \mathrm{d}\tau + \int_{-\infty}^{+\infty}\int_{-\infty}^{+\infty}w(\xi)w(\mu)\overline{f(t-\xi)f(t-\mu)}\, \mathrm{d}\xi\, \mathrm{d}\mu =
\end{equation*}

\begin{equation}
\overline{d^{2}}-2\int_{-\infty}^{+\infty} w(\tau)\rho_{fd}(\tau) \mathrm{d}\tau+\int_{-\infty}^{+\infty}\int_{-\infty}^{+\infty}w(\xi)w(\mu)\rho_{ff}(\xi-\mu) \,\mathrm{d}\xi \,\mathrm{d} \mu.
\end{equation}


Здесь используются обозначения для корреляционных функций:

\begin{equation*}
\rho_{fd}(\tau)=\overline{f(t)\,d(t+\tau)},
\end{equation*}

\begin{equation*}
\rho_{ff}(\tau)=\overline{f(t)\,f(t+\tau)}.
\end{equation*}


Черта над формулой означает осреднение по времени. Будем считать, что оптимальная импульсная характеристика системы существует и равна $w_\text{opt}$.

Тогда любая отличающаяся от нее импульсная характеристика системы может быть представлена в виде

\begin{equation}
	w(t) = w_\text{opt}(t)+\alpha\,\theta(t),
\end{equation}
где $\theta(t)$ --- произвольная функция времени, $\alpha$ --- варьируемый коэффициент.

Минимум среднеквадратической ошибки отклонения достигается при $\alpha=0$. Для поиска $w_\text{opt}(t)$ нужно найти производную показателя качества $\eta$ по коэффициенту вариации $\alpha$ и приравнять ее нулю при $\alpha=0$:

\begin{equation*}
	\frac{\partial\eta}{\partial\alpha}|_{\alpha=0} = -2\int_{-\infty}^{+\infty}\theta(\tau)\,\rho_{fd}(\tau)\, \mathrm{d}\tau +
\end{equation*}

\begin{equation*}
	\int_{-\infty}^{+\infty}\int_{-\infty}^{+\infty} \left[w_\text{opt}(\xi)\,\theta(\mu) + w_\text{opt}(\mu)\,\theta(\xi)\right] \,\rho_{ff}(\xi-\mu) \,\mathrm{d}\xi \,\mathrm{d}\mu =
\end{equation*}

\begin{equation*}
	-2\int_{-\infty}^{+\infty}\theta(\xi)\rho_{fd}(\xi)\,\mathrm{d}\xi + 2 \int_{-\infty}^{+\infty}\int_{-\infty}^{+\infty}\theta(\xi)\, w_\text{opt}(\mu)\,\rho_{ff}(\xi-\mu)\,\mathrm{d}\xi\, \mathrm{d} \mu =
\end{equation*}

\begin{equation}
	2\int_{-\infty}^{+\infty} \theta(\xi)\, \left[\int_{-\infty}^{+\infty}w_\text{opt}(\mu)\rho_{ff}(\xi-\mu)\mathrm{d}\mu- \rho_{fd}(\xi) \right]\, \mathrm{d}\xi = 0.
\end{equation}


Поскольку $\theta(\xi)$ — произвольная функция, последнее равенство выполняется тогда и только тогда, когда:

\begin{equation}
	\int_{-\infty}^{+\infty} w_\text{opt}(\mu)\, \rho_{ff}(\xi-\mu)\, \mathrm{d}\mu-\rho_{fd}(\xi)=0.
\end{equation}


Это и есть уравнение Винера-Хопфа, определяющее оптимальную импульсную характеристику системы по критерию минимальной среднеквадратической ошибки. Для решения применим преобразование Лапласа к полученному уравнению. Известно, что преобразование Лапласа от свертки равно произведению преобразований Лапласа, тогда:

\begin{equation}
	w_\text{opt}(p)S_{ff}(p)-S_{fd}(p)=0,
\end{equation}
где $w_\text{opt}(p)=L{w_\text{opt}(t)}$, $S_{ff}(p)=L{\rho_{ff}(t)}$, $S_{fd}(p)=L{\rho_{fd}(t)}$.

Таким образом определяем оптимальный винеровский фильтр 1-го рода:

\begin{equation}
	W_\text{opt I}= \frac{S_{fd}(p)}{S_{ff}(p)}
\end{equation}

Когда порядок полинома в числителе оказывается выше порядка полинома в знаменателе, винеровский фильтр 1-го рода физически нереализуем. Для решения задачи, после определения импульсной характеристики ее принудительно приравнивают нулю при отрицательных значениях  $t$ (именно отличие $w(t)$ от нуля при  $t<0$ характеризует физическую нереализуемость системы) и таким образом получают физически реализуемый винеровский фильтр 2-го рода.

\subsection{Обратная свертка Ричардсона-Люси}

По принятой модели преобразования можно записать процесс формирования сигнала изображения в виде \cite{rld}:
\begin{equation}
	g(x,y)=f(x,y)\otimes h(x,y)+n(x,y)
\end{equation}

Шумовую составляющую можно рассматривать как разность:
\begin{equation}
	n(x,y) = g(x,y) - f(x,y)\otimes h(x,y)
\end{equation}

Если вычислить достаточно точную оценку входного распределения и подставить ее в выражение, то шумовая компонента будет уменьшаться. При неточной оценке шумовая компонента будет увеличиваться. В пределе при идеальной оценке получим нулевой уровень шума.

Если в данном выражении прибавить входное распределение к левой и правой частям уравнения, то получим:

\begin{equation}
	f(x,y)=f(x,y)+[g(x,y)-f(x,y)\otimes h(x,y)]
\end{equation}

Данное уравнение может рассматриваться как итерационная процедура вычисления выходного распределения, при котором новая величина является суммой предыдущих оценок  и элемента коррекции. Элемент коррекции является разностью измеренного сигнала изображения и предыдущего вычисления с использованием текущего входного сигнала. На k-ом шаге вычислений оценка входного сигнала изображения записывается в виде:

\begin{equation}
	f_k(x,y)=f_{k-1}(x,y)+[g(x,y)-f_{k-1}(x,y)\otimes h(x,y)]
\end{equation}

Первой оценкой входного сигнала является текущее выходное распределение. Алгоритм Ричардсона-Люси итеративно вычисляет оценку максимального правдоподобия входного сигнала и реализуется в предположении, что PSF системы известно, а шум описывается распределением Пуассона. Входные и выходные сигналы представлены в виде векторов дискретных отсчетов, PSF системы - в виде двумерной матрицы. Тогда отсчеты входного, выходного сигналов и PSF системы связаны выражением дискретной свертки:

\begin{equation}\label{key}
	g_i = \sum^j h_{ij}\otimes f_j
\end{equation}

Суммирование по индексу $j$ происходит при вычислении каждого отсчета изображения с номером $i$. Тогда выражение для алгоритма Ричардсона-Люси в дискретной форме запишется в виде:

\begin{equation}
	f_j = 	f_j \sum (\frac{h_{ij}\otimes g_i}{\sum h_{jk}\otimes f_k})
\end{equation}


\section{Методы повышения разрешения}

Для решения данной задачи используются нейросети. Существует большое количество готовых обученных моделей. Следует рассмотреть несколько основных и определить наиболее подходящую:
\begin{itemize}
	\item \textbf{EDSR} --- лучшая модель с точки зрения качества выходного изображения. Однако, она также самая большая и обладает низким быстродействием \cite{edsr};
	\item \textbf{ESPCN} --- маленькая легковесная модель с высокой скоростью работы \cite{espcn};
	\item \textbf{FSRCNN} --- лучшая модель с точки зрения быстродействия, что позволяет обрабатывать видео в режиме реального времени \cite{fsrcnn};
	\item \textbf{LapSRN} --- промежуточная модель среди предложенных. Обладает средним значением скорости и качества выходного изображения \cite{lapcsrn}.
\end{itemize}







